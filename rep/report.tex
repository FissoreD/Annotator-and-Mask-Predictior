\documentclass[a4paper, 11pt]{article}

\usepackage[english]{babel}
\usepackage[utf8]{inputenc}
\usepackage[T1]{fontenc}
\usepackage{hyperref}

\begin{document}
\title{
  Annotator and Mask recognizer \\
  \large Master 2 at University Of Côte d'Azur
}
\author{Fissore Davide, Galbiati Federica and Venturelli Antoine}
\maketitle
\tableofcontents

\section{Overview}
Overview
Programming language Python version 3.9.7 with following libraries :
\begin{itemize}
  \item PIL to load and treat images
  \item tkinter to charge the graphic interface
  \item json to save and load json encoded files
  \item ttkthemes to get a larger library of theme for our interface
  \item tkhtmlview to display simple HTML and CSS text
  \item shapely to work on shapes and get coverage methods
\end{itemize}
\section{ How to run the project }
The project can be launched from the src folder (this is mandatory so that imports of local
files go correctly) opening the  window.py  file. It  will take few seconds to load all libraries, for
example the ttkthemes import is a bit slow and also the interface will be less reactive, to
avoid this you can pass the  “-fast”  optional parameter  to open the classic tk.Tk() window
(exemple  python3 ./window.py -fast,  note that the  python3 command may not work on
some OS for example in certain windows distributions you should use py).
this a subsecction
\subsection{Sub 1}
uuuf

\end{document}